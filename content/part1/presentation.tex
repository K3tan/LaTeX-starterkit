Table example : 

% Créer des tableaux LateX facilement
% http://www.tablesgenerator.com
\begin{table}[H] % Le paramètre H (majuscule) signifie : place cet élément à cet endroit précisement.
\centering
\begin{tabular}{|c|l|l|l|}
\hline
\textbf{Utilisateur} & \multicolumn{1}{c|}{\textbf{Version 1}} & \multicolumn{1}{c|}{\textbf{Version 2}} & \multicolumn{1}{c|}{\textbf{Version 3}} \\ \hline
\textit{Pierre} & Non & Non & Oui \\ \hline
\textit{Paul} & Oui & Oui & Non \\ \hline
\textit{Jacques} & Oui & Oui & Non \\ \hline
\textit{Hector} & Oui & Oui & Oui \\ \hline
\end{tabular}
\caption{Accès des différents utilisateurs aux différents versions}
\label{tab:versions}
\end{table}